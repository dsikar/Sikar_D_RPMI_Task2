\subsection{Data \& Tools}

We identified 4 image datasets, 3 environments to train and test our models, and one image labelling crowdsourcing tool, to find relevant "rainy" sections in our data.

The \textbf{Ford Multi-AV Seasonal Dataset} is a multi-sensor dataset collected by a fleet of Ford autonomous vehicles at different days and times during 2017-18. They contain inertial measurement units (IMU) that provide orientation which are our required steering angles.

The \textbf{Audi dataset} provides 40,000 frames with semantic segmentation image and point cloud labels, of which more than 12,000 frames also have annotations for 3D bounding boxes. In addition, sensor data (approx. 390,000 frames) for sequences with several loops, recorded in three cities. The data need to be evaluated for steering angle labels

The \textbf{KITTI dataset} and benchmarks for computer vision research in the context of autonomous driving. The dataset has been recorded in and around the city of Karlsruhe, Germany using the mobile platform AnnieWay and has IMU labels.

The \textbf{Udacity self-driving datasets} two datasets of interest, containing labeled images with IMU and lidar values.


We intend to locate segments where rain is present with crowdsourcing \textbf{Amazon Mechanical Turk}. The expected outcome of this step is at least dozens of sequences containing rain.

We intend to use data augmentation to create synthetic data, with properties that would emulate rain characteristics, such as superimposing rain drops to images, then adding effects such as blurring, reflection and diffusion.  
The expected outcome of this step is at least dozens of sequences containing the augmented data plus metadata indicating the level of each applied effect.

Three environments have been identified to run CNN training and testing: the \textbf{Intel DevCloud}, the \textbf{NVidia open DRIVE Constellation} platform, and the the \textbf{Udacity Self-Driving Car Simulator}. 

For initial network design we have identified \textbf{PyTorch}, \textbf{Keras} (Python Tensorflow wrapper) and \textbf{MATLAB} toolkits as potential tools.

