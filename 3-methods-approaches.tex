\section{Approaches: Methods \& Tools for Design, Analysis \& Evaluation}

An development model \parencite{Dawson:2009:PCI:1611433}, detailed description follows...

Comprehensive description of the methods used to address the question.

\subsection{mark allocation}

iii. \textbf{Approaches -- 40\%}
\textit{show how you will undertake the work to ensure that you produce meaningful results}


\subsection{Detail Task Description} 
iii. describe the methods or approaches that you will use to answer the question effectively and robustly in some detail -- you need to show that you understand what you will do in terms of any design and build or data collection and analysis;

\subsection{Proposal Structure}

A comprehensive description of the methods used to address the question. In some cases this may involve the collection of data, consideration of its nature and details describing how it would be analysed. Note that proposals must contain descriptions of planned analysis to enable the reader (including the researcher) to determine whether any data used are suitable for the task in hand. In others it may involve a description of an appropriate software design, development and evaluation methodology. In some cases both of these will be relevant. You must also be clear about how you will ensure that you are considering relevant legal and professional issues and accounting for the emotional, physical and intellectual well-being of anyone who is effected by your study -- ethical issues must be described and discussed here with any concerns identified and addressed.

Markers will look for the extent to which approaches are: comprehensively described -- they should be documented in detail* and discussed; appropriate -- they must be aligned with the question in hand and likely to result in a successful project with robust answers; informed by existing research and practice; described in a manner that enables the reader to establish the likely quality of results; indicative of deep knowledge; specific; innovative -- where innovation in terms of approach is shown to be necessary; evaluated in terms of any limitations, assumptions or issues of scope. These criteria apply to analysis, design and research ethics.

*Students often focus on data collection rather that analysis or where 'Design \& Build' is the focus on software build and technology rather than evaluation and acquisition of knowledge in their proposals. You must demonstrate that you know how you are going to establish answers to your research question(s) through your activity including any data collection or the development of (software, hardware, prototype) artefacts. These methods should be robust in terms of process and detailed in terms of their description.

\subsection{Ethical, Legal \& Professional Issues}

Discussion of the issues that are raised and ways in which they will be addressed should be included within the main body of the proposal under 'Approaches' to demonstrate capabilities in dealing with ethical issues in the RMPI assessment. 