One of the metrics used of safety is accident per million miles:

Tesla's self-reported quarterly summary statistics for vehicle safety in Q4 2019[129] reported one accident for every 3.07 million miles driven in which drivers had Autopilot engaged, compared with one accident for every 2.10 million miles driven for Tesla vehicles driven without Autopilot and without active safety features (NHTSA overall accident rate in the United States is an automobile crash every 0.479 million miles).

By 2016, data after 47 million miles of driving in Autopilot mode showed the probability of an accident was at least 50\% lower when using Autopilot.[130] During the investigation into the fatal crash of May 2016 in Williston, Florida, NHTSA released a preliminary report in January 2017 stating "the Tesla vehicles' crash rate dropped by almost 40 percent after Autosteer installation."[131][132]:10 Disputing this, in 2019, a private company, Quality Control Systems, released their report analyzing the same data, stating the NHTSA conclusion was "not well-founded".[133] Quality Control Systems' analysis of the data showed the crash rate (measured in the rate of airbag deployments per million miles of travel) actually increased from 0.76 to 1.21 after the installation of Autosteer.[134]:9

As of February 2020, Andrej Karpathy, Tesla’s head of AI and computer vision, states that: Tesla cars have driven 3 billion miles on Autopilot, of which 1 billion have been driven using Navigate on Autopilot; Tesla cars have performed 200,000 automated lane changes; 1.2 million Smart Summon sessions have been initiated.[135] He also states that Tesla cars are avoiding pedestrian accidents at a rate of tens to hundreds per day.[136]