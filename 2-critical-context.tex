\section{Critical Context - literature review?}

\subsection{Mark Allocation}

ii. \textbf{Critical Context -- 15\%}

\textit{show that your ideas are informed by relevant information}

\subsection{Detail Task Description} 
iv. \textit{critically evaluate} the methods and likely results in the context of the question and the application domain showing that you have an understanding of the scope and reliability of the results that the planned work will produce -- it is important to demonstrate self-awareness here;

viii. ensure that the work is informed by a survey and synthesis of relevant academic literature and appropriate technical documentation where relevant to establish critical context;

\subsection{Proposal Structure}
A critical description of the context in which the work will take place. A review of no fewer than 5 relevant documents should help you define and contextualise the research question and inform the subsequent choice of methods to be used. Academic literature should be used and sources cited throughout the proposal to inform the work and justify the statements that are made as the arguments develop.

\textbf{Markers will look for: the extent to which appropriate literature has been identified and consulted, this work is used to inform the project with coherent arguments developed to justify and inform the proposal; levels of engagement with the literature; a comprehensive and systematic approach to searching the literature; whether references are evaluated and whether high quality academic references are used*; whether the literature is understood, synthesised and applied to the task hand; the sophistication of arguments used; originality in the arguments presented; the currency and relevance of sources used; important references or sources that are omitted.}

\textbf{*Note that broader sources, such as technical documentation, descriptions of competing products, applicable law, relevant software, text books and research methods literature may also be relevant -- as discussed with your supervisor.}

\subsection{Another Subsection}

No fewer than 5 references



Examples of lists:
\begin{itemize}
\item Bullet point one
\item Bullet point two
\end{itemize}

\begin{enumerate}
\item Numbered list item one
\item Numbered list item two
\end{enumerate}

\textbf{Example of table reference: see Table \ref{tab:example}}.
\lipsum[4]

\begin{table}[ht] 
\centering
\begin{tabular}{l l l}
\hline
\textbf{Treatments} & \textbf{Response 1} & \textbf{Response 2}\\
\hline
Treatment 1 & 0.0003262 & 0.562 \\
Treatment 2 & 0.0015681 & 0.910 \\
Treatment 3 & 0.0009271 & 0.296 \\
\hline
\end{tabular}
\caption{Table caption}
\label{tab:example}
\end{table}

More stuff to consider \lipsum[1]