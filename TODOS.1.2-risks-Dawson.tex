\textbf{Atkins}: Using Dawson’s (2006) framework for risk management, I have identified a set of potential risks for my project. I have outlined the nature of the
risk, its risk likelihood, risk consequence, risk impact (likelihood * consequence). I describe the mitigating steps I will take to reduce either the
likelihood or consequence of these risks as well as as any contingency plan if the risk eventuates. 

\textbf{Dimmock}: No text  

\textbf{Manchev}: The risk assessment and management is based on the strategy outlined by Dawson[29].
The key risks associated with the research are outlined in Table 1. We have listed a combination
of external and internal risks and the corresponding alleviation strategies.
We intend to keep the risk register up to date by adding, removing, and re-evaluating
individual risks as the project progresses.

\textbf{Minah}: The following risk register outlines the envisaged risks, their associated likelihoods, consequences and impacts, and what has been done or will be done to mitigate them. 

\textbf{Walters}: Risk management includes a quantitative assessment of any risks that may occur during the
project. The risk table will be updated throughout the project, adding any new risks that may
develop as well as removing risks that are no longer a threat. The likelihood of the risk will be
estimated between 1 and 3 and the consequence between 1 and 5, with the impact being the
multiple of these two (Dawson, n.d.). The response provides a solution if these risks were to
happen.  

\textbf{Wells}: No text