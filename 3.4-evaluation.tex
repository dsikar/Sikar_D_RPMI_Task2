%% \subsubsection{Evaluation}

\subsection{Evaluation}

\textbf{This paragraph might be best placed in methods section} Our data consists of videos, taken by cameras mounted on moving vehicles, which can be interpreted as sequences of still images taken at fixed-time intervals. Each still image is labelled with a quantity we call $\theta$, that represents a direction from 0 to 359 degrees, in which the vehicle was travelling at the moment the image was stored. $\Delta \theta$ between two images, taken at intervals $i$ and $i+n$, represents the amount of steering that was applied to the vehicle after n intervals. Therefore the assumption is we are dealing with a regression problem, where, given a sequence of images as described, we want to approximate $\theta$ to keep the autonomous vehicle on the road.


Developing the evaluation metric is part of this project. We are specifically interested in the steering aspect, and how much error (oversteering or understeering) would define an accident.

As such, the initial proposed evaluation metric is to, once the networks have been trained and tested on our original "dry" data, obtain a ground truth, where no accidents (autonomous vehicles driving off the road) are registered. We will then use our "rainy" and augmented data. The expectation is that at a given threshold, the autonomous network under test will output steering angles for a sufficient amount of frames that would cause the car to steer off the road.
Once a few of these cases have been identified, the sequences would be used for all networks, and an evaluation made. The artifact being a table with rows containing entries for networks and conditions that caused an accident, plus a discussion of the results, which would then \textbf{answer our research question} on \textbf{evaluation of self-driving cars using CNNs in the rain}. 